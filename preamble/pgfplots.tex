\usepackage{pgfplots}%
\pgfplotsset{compat=1.17}

\usepackage{fp} % 高精

\def\pitypesetter#1{%
	\pgfmathparse{#1}%
	\ifdim\pgfmathresult pt=0.0pt%
		%
	\else%
		\pgfmathparse{#1/\FPpi}%
		\ifdim\pgfmathresult pt=1.0pt%
			$\pi$%
		\else%
			\ifdim\pgfmathresult pt=-1.0pt%
				$-\pi$%
			\else%
				\pgfmathprintnumber{\pgfmathresult}$\pi$%
			\fi%
		\fi%
	\fi%
}

\pgfplotsset{
	% Typeset fractions of pi at regular intervals on x axis
	pi denominator/.style={
		% Make sure the x axis is in radians
		trig format plots=rad,
		% Set tick distance from style argument
		xtick distance={\FPpi/#1},
		% Set label style: calculate reduced fraction of pi
		xticklabel={
			\tikzmath{
				% Calculate this tick's multiple of pi/#1
				int \numorig, \numgcd, \num, \denom, \absnum;
				\numorig = round(\tick*#1/pi);
				% Calculate reduced fraction for \numorig/#1
				\numgcd = gcd(\numorig,#1);
				\num = \numorig / \numgcd;
				\absnum = abs(\num);
				\denom = #1 / \numgcd;
				% Build label text
				if \num < 0 then {
						let \sign = -;
				} else {
						let \sign =;
				};
				if \absnum == 1 then {
						let \numpi = \pi;
				} else {
						let \numpi = \absnum\pi;
				};
				if \denom == 1 then {
					if \num == 0 then {
							{ \strut$0$ };
					} else {
							{ \strut$\sign\numpi$ };
					};
				} else {
					{ \strut$\sign\frac{\numpi}{\denom}$ };
					% Other style with all pi symbols same and aligned:
					%{ \strut$\sign\frac{\absnum}{\denom}\pi$ };
				};
			}
		},
	},
	pi denominator/.default=2,
	math textstyle/.code={\everymath{\textstyle}},
}

\pgfplotsset{
	my axis/.style={
		baseline,
		axis lines=middle,
		x=1cm,
		y=1cm,
		xtick distance=1,
		ytick distance=1,
		restrict y to domain=-4:4,
		enlargelimits={abs=0.5},
		grid=major,
		legend style={draw=none},
		legend pos=outer north east,
		axis line style={->},
	},
}

\pgfplotsset{
	integral plot/.is family,
	integral plot,
	default/.style={
		integrand/.value required,
		integral/.value required,
		domain=-4:4,
		samples=33,
		smooth=true,
	},
	% TODO: these keys are passed in a strange way, who can fix it?
	integrand/.store in=\integrand,
	integral/.store in=\integral,
	domain/.store in=\domain,
	samples/.store in=\samples,
	smooth/.store in=\smooth,
	% collect unknown keys in style 'remainingkeys':
	.unknown/.code={%
		\let\currname\pgfkeyscurrentname%
		\let\currval\pgfkeyscurrentvalue%
		\ifx#1\pgfkeysnovalue%
			\pgfqkeys{/pgfplots}{remainingkeys/.append style/.expand once={\currname}}%
		\else%
			\pgfqkeys{/pgfplots}{remainingkeys/.append style/.expand twice={\expandafter\currname\expandafter=\currval}}%
		\fi%
	},%
}

\pgfplotsset{
	function generated by two plot/.is family,
	function generated by two plot,
	default/.style={
		first/.value required,
		second/.value required,
		final/.value required,
		domain=-4:4,
		samples=33,
		smooth=true,
	},
	% TODO: these keys are passed in a strange way, who can fix it?
	first/.store in=\first,
	second/.store in=\second,
	final/.store in=\final,
	domain/.store in=\domain,
	samples/.store in=\samples,
	smooth/.store in=\smooth,
	% collect unknown keys in style 'remainingkeys':
	.unknown/.code={%
		\let\currname\pgfkeyscurrentname%
		\let\currval\pgfkeyscurrentvalue%
		\ifx#1\pgfkeysnovalue%
			\pgfqkeys{/pgfplots}{remainingkeys/.append style/.expand once={\currname}}%
		\else%
			\pgfqkeys{/pgfplots}{remainingkeys/.append style/.expand twice={\expandafter\currname\expandafter=\currval}}%
		\fi%
	},%
}

\pgfplotsset{
	regular axis/.style={
		my axis,
		math textstyle,
	},
	pi axis/.style={
		regular axis,
		pi denominator,
	},
}

\pgfplotscreateplotcyclelist{function generated by two}{
	{magenta},{cyan},{thick}
}

\newcommand{\integralplot}[1][]{
	\pgfqkeys{/pgfplots}{remainingkeys/.style={}} % reset remainingkeys
	\pgfqkeys{/pgfplots/integral plot}{default, #1}
	\addplot [
		domain=\domain,
		samples=\samples,
		smooth=\smooth,
		/pgfplots/remainingkeys,
	]
	{\integrand};
	\addplot [
		domain=\domain,
		samples=\samples,
		smooth=\smooth,
		teal,
		thick,
		/pgfplots/remainingkeys,
	]
	{\integral};
	\addplot [
		domain=\domain,
		samples=\samples,
		smooth=\smooth,
		draw=none,
		fill=teal,
		fill opacity=.125,
		/pgfplots/remainingkeys,
	]
	{\integrand} \closedcycle;
}

\newcommand{\funcgenbytwoplot}[1][]{
	\pgfqkeys{/pgfplots}{remainingkeys/.style={}} % reset remainingkeys
	\pgfqkeys{/pgfplots/function generated by two plot}{default, #1}
	\addplot+ [
		domain=\domain,
		samples=\samples,
		smooth=\smooth,
		/pgfplots/remainingkeys,
	]
	{\first};
	\addplot+ [
		domain=\domain,
		samples=\samples,
		smooth=\smooth,
		/pgfplots/remainingkeys,
	]
	{\second};
	\addplot+ [
		domain=\domain,
		samples=\samples,
		smooth=\smooth,
		/pgfplots/remainingkeys,
	]
	{\final};
}