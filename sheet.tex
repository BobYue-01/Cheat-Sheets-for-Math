\documentclass[twoside]{report}
\usepackage{ctex}
	\setCJKmainfont{方正书宋_GBK.TTF}[BoldFont = {方正小标宋_GBK.TTF}, ItalicFont = {方正楷体_GBK.TTF}] 
	\setCJKsansfont{方正黑体_GBK.TTF}[BoldFont = {方正大黑_GBK.TTF}] 
	\setCJKmonofont{方正仿宋_GBK.TTF}
\usepackage{amssymb}
\usepackage{amsmath}
\usepackage{autobreak}
\usepackage{indentfirst}
	\setlength{\parindent}{2em}
\usepackage{amsthm}
\usepackage{enumitem}
\usepackage[colorlinks,linkcolor=blue,anchorcolor=red,citecolor=green]{hyperref}
\usepackage[left=1in,right=1in,top=1in,bottom=1in]{geometry}
\usepackage{tcolorbox}
% Workaround for gcd() issue in pgfplots 1.14
% (see https://sourceforge.net/p/pgfplots/bugs/129/ and
% https://tex.stackexchange.com/questions/328972/ )
\usepackage{tikz}
\makeatletter
\let\pgfmathgcdX=\pgfmathgcd@
\usepackage{pgfplots}%
\let\pgfmathgcd@=\pgfmathgcdX
\makeatother
\usepackage{etex}
\usepackage{fp} % 高精
\usetikzlibrary{datavisualization}
\usetikzlibrary{datavisualization.formats.functions}
\usetikzlibrary{fixedpointarithmetic}
\usetikzlibrary{math}
\FPdiv \halfpi \FPpi 2 % 高精
\def\pitypesetter#1{%
	\pgfmathparse{#1}%
	\ifdim\pgfmathresult pt=0.0pt%
		%
	\else%
		\pgfmathparse{#1/\FPpi}%
		\ifdim\pgfmathresult pt=1.0pt%
			$\pi$%
		\else%
			\ifdim\pgfmathresult pt=-1.0pt%
				$-\pi$%
			\else%
				\pgfmathprintnumber{\pgfmathresult}$\pi$%
			\fi%
		\fi%
	\fi%
}
\makeatletter
\pgfplotsset{
	% Typeset fractions of pi at regular intervals on x axis
	x axis in pi/.style={
		% Make sure the x axis is in radians
		trig format plots=rad,
		% Set tick distance from style argument
		xtick distance={pi/#1},
		% Set label style: calculate reduced fraction of pi
		xticklabel={
			\tikzmath{
				% Calculate this tick's multiple of pi/#1
				int \numorig, \gcd, \num, \denom, \absnum;
				\numorig = round(\tick*#1/pi);
				% Calculate reduced fraction for \numorig/#1
				\gcd = gcd(\numorig,#1);
				\num = \numorig / \gcd;
				\absnum = abs(\num);
				\denom = #1 / \gcd;
				% Build label text
				if \num < 0 then {
						let \sign = -;
				} else {
						let \sign =;
				};
				if \absnum == 1 then {
						let \numpi = \pi;
				} else {
						let \numpi = \absnum\pi;
				};
				if \denom == 1 then {
					if \num == 0 then {
							{ \strut$0$ };
					} else {
							{ \strut$\sign\numpi$ };
					};
				} else {
					{ \strut$\sign\frac{\numpi}{\denom}$ };
					% Other style with all pi symbols same and aligned:
					%{ \strut$\sign\frac{\absnum}{\denom}\pi$ };
				};
			}
		},
},
}
\renewenvironment{proof}[1][\proofname]{\par
	\pushQED{\qed}
	\normalfont \topsep6\p@\@plus6\p@\relax
	\begin{itemize}[
		leftmargin    = 0em,
		itemindent    = 2em,
		labelsep      = 1em,
		listparindent = 2em
	]
		\item[\bfseries#1]
}{
	\popQED \end{itemize} \@endpefalse
}
\makeatother
\renewcommand{\proofname}{证}
\newenvironment{solution}{\begin{proof}[解]}{\end{proof}}
\newenvironment{analysis}{\begin{proof}[析]\let\qed\relax}{\end{proof}}
\newenvironment{innerprf}{\begin{proof}[析]\let\qed\relax}{\end{proof}}
\newcounter{question}[chapter]
\def\questionautorefname{题}
\newenvironment{question}[3][]{
	\begin{enumerate}[
			labelsep=2em,
			ref=\thequestion
		]
		\begin{tcolorbox}[
				title=\textsf{#2},
				box align=top
			]
		\refstepcounter{question}
		\item[\bfseries\thequestion.]{#1}
			{#3}
		\end{tcolorbox}
		\nopagebreak
}{
	\end{enumerate}
}
\numberwithin{equation}{question}
\renewcommand{\theequation}{\arabic{question}.\arabic{equation}}
\newcommand{\source}[1]{\begin{flushright}(\textit{#1})\end{flushright}}
\theoremstyle{definition}\newtheorem{theorem}{定理}
\theoremstyle{definition}\newtheorem{axiom}[theorem]{公理}
\theoremstyle{definition}\newtheorem{rules}[theorem]{规则}
\theoremstyle{definition}\newtheorem{corollary}[theorem]{推论}
\newcommand{\dd}{\ \text{d}}
\newcommand{\axisoption}{
	baseline,
	axis lines=middle,
	x=1cm,
	y=1cm,
	% axis equal image,
	restrict y to domain=-5:5,
	enlargelimits={abs=0.5},
	grid=major,
	legend style={draw=none},
	legend pos=outer north east,
	axis line style={->},
}
\newcommand{\trigonometricplotpreprocess}[2]{
	\pgfmathsetmacro{\minpi}{round(4*#1/\FPpi)}
	\pgfmathsetmacro{\maxpi}{round(4*#2/\FPpi)}
	\foreach \X [count=\Y] in {\minpi,...,\maxpi} {
		\pgfmathsetmacro{\myx}{\X*\FPpi/4}
		\ifnum\Y=1
			\xdef\LstX{\myx}
		\else
			\xdef\LstX{\LstX,\myx}
		\fi
	}
}
\newenvironment{regularplot}[2]{
	\begin{tikzpicture}
		\everymath{\textstyle}
		\begin{axis}[
			\axisoption
		]
}{
		\end{axis}
		\everymath{\displaystyle}
	\end{tikzpicture}
}
\newenvironment{trigonometricplot}[2]{
	\trigonometricplotpreprocess{#1}{#2}
	\begin{tikzpicture}
		\everymath{\textstyle}
		\begin{axis}[
			x axis in pi=2,
			\axisoption
		]
}{
		\end{axis}
		\everymath{\displaystyle}
	\end{tikzpicture}
}
\newcommand{\integralplot}[4]{
	\addplot [
			smooth,
			domain=#1:#2,
			samples=16*2+1,
		]
		{#3};
		\addplot [
			smooth,
			teal,
			thick,
			domain=#1:#2,
			samples=16*2+1,
		]
		{#4};
		\addplot [
			smooth,
			draw=none,
			fill=teal,
			fill opacity=.125,
			domain=#1:#2,
			samples=16*2+1,
		]
		{#3} \closedcycle;
}
\newcommand{\regularintegral}[6]{
	\begin{regularplot}{#1}{#2}
		\integralplot{#1}{#2}{#3}{#4}
		\legend{#5,#6};
	\end{regularplot}
}
\newcommand{\trigonometricintegral}[6]{
	\begin{trigonometricplot}{#1}{#2}
		\integralplot{#1}{#2}{#3}{#4}
		\legend{#5,#6};
	\end{trigonometricplot}
}
\newcommand{\trigonometricintegraltworanges}[8]{
	\begin{trigonometricplot}{#1}{#4}
		\integralplot{#1}{#2}{#5}{#6}
		\integralplot{#3}{#4}{#5}{#6}
		\legend{#7,#8};
	\end{trigonometricplot}
}

\usepackage{fancyhdr}
\fancypagestyle{plain}
{
	\fancyhf{}
	\fancyfoot[RO]{\bfseries\thepage}
	\fancyfoot[LE]{\bfseries\thepage}
	\renewcommand{\headrulewidth}{0pt}
}
\pagestyle{fancy}
\renewcommand{\chaptermark}[1]{\markboth{#1}{}}
\renewcommand{\sectionmark}[1]{\markright{\thesection\ #1}}
\fancyhf{}
\fancyfoot[RO]{\bfseries\thepage}
\fancyfoot[LE]{\bfseries\thepage}
\fancyhead[LE]{\bfseries\leftmark}
\fancyhead[RE]{\bfseries 数学速查表\\Cheat Sheets for Math}
\fancyhead[LO]{\bfseries 红雀\\Cardinalis}
\fancyhead[RO]{\bfseries\rightmark}
\renewcommand{\headrulewidth}{0.4pt}
\renewcommand{\footrulewidth}{0pt}

\everymath{\displaystyle}
\begin{document}

\begin{titlepage}
	\centering
	\rule{\textwidth}{1pt}

	\vspace{0.15\textheight}

	{\Huge \bfseries 数学速查表\\\vspace{1ex}Cheat Sheets for Math}

	\vspace{0.025\textheight}

	\rule{0.8\textwidth}{0.4pt}

	\vspace{0.1\textheight}

	{\Large 红雀\\\vspace{1ex}Cardinalis}

	\vfill

	{\large \today}

	\vspace{0.1\textheight}

	\rule{\textwidth}{1pt}
\end{titlepage}

\tableofcontents

\setcounter{chapter}{-1}

\chapter{序\\Prologue}

\chapter{积分表\\Table of Integrals}

\section{三角函数\\Trigonometric Functions}


\begin{question}{}
	{
		$\int \sin x \dd x = -\cos x + C$
	}
	\trigonometricintegral
		{-\FPpi}{\FPpi}
		{sin(x)}{-cos(x)}
		{$\sin x$}{$-\cos x$}
\end{question}

\begin{question}{}
	{
		$\int \cos x \dd x = \sin x + C$
	}
	\trigonometricintegral
		{-\FPpi}{\FPpi}
		{cos(x)}{sin(x)}
		{$\cos x$}{$\sin x$}
\end{question}

\begin{question}{}
	{
		$\int \sec^2 x \dd x = \tan x + C$
	}
	\begin{trigonometricplot}{-\FPpi/2}{3*\FPpi/2}
		\addplot [
			smooth,
			magenta,
			domain=-\FPpi/2:3*\FPpi/2,
			samples=16*2+1,
		]
		{cos(x)};
		\addplot [
			smooth,
			cyan,
			domain=-\FPpi/2+0.001:\FPpi/2-0.001,
			samples=16*2+1,
		]
		{sec(x)};
		\addplot [
			smooth,
			thick,
			domain=-\FPpi/2+0.001:\FPpi/2-0.001,
			samples=16*2+1,
		]
		{(sec(x))^2};
		\addplot [
			smooth,
			cyan,
			domain=\FPpi/2+0.001:3*\FPpi/2-0.001,
			samples=16*2+1,
		]
		{sec(x)};
		\addplot [
			smooth,
			thick,
			domain=\FPpi/2+0.001:3*\FPpi/2-0.001,
			samples=16*2+1,
		]
		{(sec(x))^2};
		\legend{$\cos x$,$\sec x$,$\sec^2 x$};
	\end{trigonometricplot}
	\par
	\trigonometricintegral
		{-\FPpi/2+0.001}{\FPpi/2-0.001}
		{(sec(x))^2}{tan(x)}
		{$\sec^2 x$}{$\tan x$}
\end{question}

\begin{question}{}
	{
		$\int \csc^2 x \dd x = -\cot x + C$
	}
	\begin{trigonometricplot}{-\FPpi}{\FPpi}
		\addplot [
			smooth,
			magenta,
			domain=-\FPpi:\FPpi,
			samples=16*2+1,
		]
		{sin(x)};
		\addplot [
			smooth,
			cyan,
			domain=-\FPpi+0.001:-0.001,
			samples=16*2+1,
		]
		{1/sin(x)};
		\addplot [
			smooth,
			thick,
			domain=-\FPpi+0.001:-0.001,
			samples=16*2+1,
		]
		{(1/sin(x))^2};
		\addplot [
			smooth,
			cyan,
			domain=0.001:\FPpi-0.001,
			samples=16*2+1,
		]
		{1/sin(x)};
		\addplot [
			smooth,
			thick,
			domain=0.001:\FPpi-0.001,
			samples=16*2+1,
		]
		{(1/sin(x))^2};
		\legend{$\sin x$,$\csc x$,$\csc^2 x$};
	\end{trigonometricplot}
	\par
	\trigonometricintegral
		{0.001}{\FPpi-0.001}
		{(1/sin(x))^2}{-cot(x)}
		{$\csc^2 x$}{$-\cot x$}
\end{question}

\begin{question}{}
	{
		$\int \sec x \tan x \dd x = \sec x + C$
	}
	\begin{trigonometricplot}{-\FPpi/2}{3*\FPpi/2}
		\addplot [
			smooth,
			magenta,
			domain=-\FPpi/2+0.001:\FPpi/2-0.001,
			samples=16*2+1,
		]
		{sec(x)};
		\addplot [
			smooth,
			cyan,
			domain=-\FPpi/2+0.001:\FPpi/2-0.001,
			samples=16*2+1,
		]
		{tan(x)};
		\addplot [
			smooth,
			thick,
			domain=-\FPpi/2+0.001:\FPpi/2-0.001,
			samples=16*2+1,
		]
		{sec(x)*tan(x)};
		\addplot [
			smooth,
			magenta,
			domain=\FPpi/2+0.001:3*\FPpi/2-0.001,
			samples=16*2+1,
		]
		{sec(x)};
		\addplot [
			smooth,
			cyan,
			domain=\FPpi/2+0.001:3*\FPpi/2-0.001,
			samples=16*2+1,
		]
		{tan(x)};
		\addplot [
			smooth,
			thick,
			domain=\FPpi/2+0.001:3*\FPpi/2-0.001,
			samples=16*2+1,
		]
		{sec(x)*tan(x)};
		\legend{$\sec x$,$\tan x$,$\sec x\tan x$};
	\end{trigonometricplot}
	\par
	\trigonometricintegraltworanges
		{-\FPpi/2+0.001}{\FPpi/2-0.001}
		{\FPpi/2+0.001}{3*\FPpi/2-0.001}
		{sec(x)*tan(x)}{sec(x)}
		{$\sec x\tan x$}{$\sec x$}
\end{question}

\begin{question}{}
	{
		$\int \csc x \cot x \dd x = -\csc x + C$
	}
	\begin{trigonometricplot}{-\FPpi}{\FPpi}
		\addplot [
			smooth,
			magenta,
			domain=-\FPpi+0.001:-0.001,
			samples=16*2+1,
		]
		{1/sin(x)};
		\addplot [
			smooth,
			cyan,
			domain=-\FPpi+0.001:-0.001,
			samples=16*2+1,
		]
		{cot(x)};
		\addplot [
			smooth,
			thick,
			domain=-\FPpi+0.001:-0.001,
			samples=16*2+1,
		]
		{1/sin(x)*cot(x)};
		\addplot [
			smooth,
			magenta,
			domain=0.001:\FPpi-0.001,
			samples=16*2+1,
		]
		{1/sin(x)};
		\addplot [
			smooth,
			cyan,
			domain=0.001:\FPpi-0.001,
			samples=16*2+1,
		]
		{cot(x)};
		\addplot [
			smooth,
			thick,
			domain=0.001:\FPpi-0.001,
			samples=16*2+1,
		]
		{1/sin(x)*cot(x)};
		\legend{$\sec x$,$\tan x$,$\sec x\tan x$};
	\end{trigonometricplot}
	\par
	\trigonometricintegraltworanges
		{-\FPpi+0.001}{-0.001}
		{0.001}{\FPpi-0.001}
		{1/sin(x)*cot(x)}{-1/sin(x)}
		{$\csc x\cot x$}{$-\csc x$}
\end{question}

\begin{question}{}
	{
		$\int \cosh x \dd x = \sinh x + C$
	}
	\regularintegral
		{-5}{5}
		{cosh(x)}{sinh(x)}
		{$\cosh x$}{$\sinh x$}
\end{question}

\begin{question}{}
	{
		$\int \sinh x \dd x = \cosh x + C$
	}
	\regularintegral
		{-5}{5}
		{sinh(x)}{cosh(x)}
		{$\sinh x$}{$\cosh x$}
\end{question}

\begin{question}{}
	{
		$\int \frac{1}{\cosh^2 x} \dd x = \tanh x + C$
	}
	\regularintegral
		{-5}{5}
		{1/(cosh(x))^2}{tanh(x)}
		{$\frac{1}{\cosh^2 x}$}{$\tanh x$}
\end{question}

\end{document}